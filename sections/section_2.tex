\section{Основные составляющие защиты информации.}
\subsection{Виды угроз и способы защиты.}
Появление новых информационных технологий и развитие мощных компьютерных систем хранения и обработки информации повысили уровни защиты информации и вызвали 
необходимость в том, чтобы эффективность защиты информации росла вместе со сложностью архитектуры хранения данных. Так постепенно защита экономической информации 
становится обязательной: разрабатываются всевозможные документы по защите информации; формируются рекомендации по защите информации; даже проводится федеральный закон 
о защите информации. В правовых основах защиты информации можно выделить 4 основных уровня:
\begin{itemize}
    \item \textbf{Первый уровень.} Состоит из из международных договоров о защите информации и государственной тайны;
    \item \textbf{Второй уровень.} Здесь находятся подзаконные акты "--- указы президента, постановления Правительства и т.д;
    \item \textbf{Третий уровень.} К данному уровню обеспечения правовой защиты информации относятся ГОСТы безопасности информационных технологий, а также обеспечения безопасности информационных систем.
    Также на третьем уровне безопасности информационных технологий присутствуют руководящие документы, нормы информационной безопасности и классификаторы, разрабатывающиеся 
    государственными органами;
    \item \textbf{Четвёртый уровень.} Образуют локальные нормативные акты, инструкции и положения по комплексной правовой защите\cite{def_inf}.
\end{itemize}
Виды угроз и атак также делят на несколько категорий. Среди категорий выделяют:
\begin{itemize}
    \item Перехват электронных излучений. Проблема решается обеспечением защиты информации, передаваемой по радиоканалам связи и обмена данными 
    информационной системы;
    \item Применение подслушивающих устройств;
    \item Использование недостатков языков программирования с целью маскирования под запросы системы;
    \item Злоумышленный вывод из строя технологий и механизмов;
    \item Информационные утечки.
\end{itemize}

У всех вышеперечисленных угроз одна общая цель, а именно "--- призвать утечку информации юридического лица, кампании.

\newpage
\subsection{Утечки информации и их последствия.}
С увеличением масштабов распространения и использования ЭВМ и информационных сетей усиливается роль различных факторов, вызывающих утечку, 
разглашение и несанкционированный доступ к информации. К ним относятся:
\begin{itemize}
    \item Ошибки пользователей и персонала;
    \item Технические сбои;
    \item Природный фактор (природные явления, аварии).
\end{itemize}
В наши дни количество зарегистрированных случаев утечек информации с каждым годом только растёт. Утечка происходит как и намеренно, с целью получения
материальной выгоды, так и случайно, например, из"=за несоблюдения персоналом правил информационной безопасности. Именно на этой основе все утечки фирмы или кампании
делятся на \textbf{умышленные} и \textbf{неумышленные.} К сожалению, в силу халатности и неправомерных действий сотрудников, нередки случаи утечки информации или целых
баз данных.

\subsection{Принципы информационной безопасности.}
Существует 3 основных принципа информационной безопасности, которые основываются на свойствах информации: конфиденциальность, целостность и доступность информации.

\textbf{Принцип конфиденциальности} постулирует необходимость обеспечения возможности получения передаваемой информации только адресатом. Из этого вытекает, что передаваемая информация
не должна быть получена \textbf{и считана} третьими лицами, в частности злоумышленниками.

\textbf{Принцип целостности} заключается в корректности и неизменности передаваемой информации. За время передачи информации получателю, информация должна оставаться полной и актуальной.
Помимо этого, информация должна сохранять первозданный вид, несмотря на внешние искажения.

\textbf{Принцип доступности} говорит о том, что доступ к передаваемой информации должны иметь только легитимные пользователи, то есть только те пользователи, которым изначально
передавалась информация.

Осуществление и выполнение подобных принципов достигается различными методами защиты от внеших угроз, о которых будет сказано ниже.


\newpage
\subsection{Классификация средств защиты информации. Методы защиты данных. Шифрование.}
В своей основе, средства защиты информации классифицируются на \textbf{формальные} и \textbf{неформальные.} Формальные средства защиты информации подразумевают исполнение процедур по защите 
без вмешательства человека, в то время как неформальные "--- полностью основанны на действиях человека.

Среди основных методов защиты информации выделяют:
\begin{itemize}
    \item \textbf{Криптографические} "--- защита данных, передаваемых по глобальной или корпоративной сети. Включает в себя различные типы шифрования, в частности с использованием специальных аппаратов
    и программ;
    \item \textbf{Физические} "--- средства и механизмы, работающие вне зависимости от информационных систем;
    \item \textbf{Программные} "--- защита передаваемой информации посредством специального ПО, настроенного под конкретную сеть и конкретные задачи.
\end{itemize}

В своей сути у большинства метода заложен различный метод \textbf{шифрования.} Шифрование "--- это процесс преобразования информации путём исполнения конкретного алгоритма с целью сохранения конфиденциальности
данной информации. Существует также процесс, обратный шифрованию "--- \textbf{дешифрование.} Преобразование производится согласно выбранному алгоритму шифрования 
и, отталкиваясь от этого, выполняется определённая процедура.

\textbf{Криптография} предлагает множество вариантов шифрования, и, соответственно, дешифрования, однако к классическим методам принято относить \textbf{симметричное шифрование.} Это тип шифрования, в котором и для шифрования, и для дешифрования,
используется единый криптографический ключ. Основным условием симметричного шифрования является факт того, что отправитель и получатель информации заранее знают о 
методе шифрования и заранее имеют на руках криптографический ключ, который необходимо держать в секрете с обеих сторон. Однако у симметричного шифрования существует ряд недостатков: наличие ключа как у отправителя, так и у получателя.
Из этого следует, что изначально сам ключ должен быть передан безопасным путём. Решает эти проблемы не менее популярное \textbf{ассиметричное шифрование,} при котором для шифрования и для шифрования используются, соответственно, два разных ключа.
Так, ключ для дешифрования генерируется у получателя, которому достаточно знать только метод шифрования\cite{shifr}.





