\section{Профессиональное развитие и карьерные перспективы.}
\subsection{Кто такой специалист по защите информации сейчас.}
\textbf{Специалист по защите информации} "---  это высококвалифицированный специалист, который ответственен за защиту конфиденциальных 
данных, информационных систем и сетей от утечек, кражи и других видов злоупотреблений. С увеличением объема и ценности 
информации, хранимой и передаваемой в цифровом формате, растет и важность профессионального защитника информации.

Основные обязанности специалиста по защите информации включают разработку и внедрение мер безопасности информационных систем, 
анализ уязвимостей, прогнозирование и предотвращение кибератак, обучение сотрудников правилам безопасности. Также в их 
компетенции находится расследование инцидентов безопасности, разработка политики безопасности и соблюдение соответствующих 
законов и стандартов.
Но есть и более узкие специальности уже внутри сферы:
\begin{itemize}
    \item \textbf{Пентестеры} "---  так называемые «белые» или «этичные» хакеры. Они не взламывают ресурсы бизнеса незаконно. Вместо 
    этого они работают на компании и ищут уязвимости, которые потом исправляют разработчики. Бывает, такие люди трудятся на 
    окладе, или участвуют в программах Bug Bounty "---  когда бизнес просит проверить их защиту, обещая за найденные баги премию.
    \item \textbf{Специалисты по разработке} "--- такие специалисты участвуют создании приложений и программ. Проще говоря, 
    изучают архитектуру и готовый код и подсказывают, что здесь может быть ошибка или «форточка» для взлома. Банальный пример 
    "---  оставить в форме ввода сайта возможность отправить SQL-инъекцию.
    \item \textbf{Специалисты по сетям} "---  они ищут возможные потенциальные и известные уязвимости в аппаратных и сетевых 
    комплексах. Проще говоря, знают, как с помощью Windows, Linux или других систем злоумышленник может попасть в ваш 
    компьютер и установить нужное ПО. Могут как найти возможность взлома, так и создать систему, в которую будет сложно 
    попасть.
\end{itemize}
Есть ещё один вариант деления специалистов:
\begin{itemize}
    \item Отвечающие за взлом, к примеру, сети или программы. Иначе называется этичный хакинг "---  специализация на обнаружении ошибок и уязвимостей.
    \item Отвечающие за создание и поддержку систем защиты. Этот вариант работодатели подразумевают, когда ищут специалиста по защите информации\cite{habr}.
    
\end{itemize}

Для успешной работы специалист по защите информации должен обладать знаниями в области информационных технологий, 
криптографии, сетевой безопасности, законодательства о защите данных. Он также должен иметь навыки аналитического мышления, 
умение принимать быстрые и правильные решения в экстренных ситуациях, уметь анализировать риски информационной безопасности и 
предлагать эффективные меры по их минимизации, а также быть готовым к постоянному обучению и развитию.

В современном мире спрос на специалистов по защите информации растет, так как компании и организации всё чаще сталкиваются с 
киберугрозами, взломами и другими атаками на информационную инфраструктуру. Поэтому специалисты по защите информации являются 
востребованными специалистами на рынке труда и обладают хорошими перспективами карьерного роста.
Из-за неустоявшихся терминов есть небольшая путаница и в названиях вакансий — компании ищут специалистов по информационной 
безопасности, администраторов защиты, инженеров безопасности компьютерных сетей и другие названия, подразумевая одного и того 
же специалиста.
\newpage
\subsection{Типы <<хакеров>>: черные, белые, серые.}
Если вы смотрите новости и следите за технологиями, вы знаете, кто такие хакеры. Однако не все знают, что хакеры делятся на категории, называемые Черные шляпы, Белые шляпы и Серые шляпы. Эти термины берут свое начало в американских вестернах, где главные герои носили белые или светлые шляпы, а отрицательные персонажи "---  черные шляпы.
По сути, тип хакера определяется его мотивацией и тем, нарушает ли он закон.

\textbf{Кто же такие <<Чёрные шляпы>>?}
«Чёрные шляпы» "---  это хакеры, которые используют свои навыки для незаконного проникновения в компьютерные системы и кражи данных. Они могут быть как начинающими дилетантами, так и опытными профессионалами, работающими на крупные преступные организации.

Хакерство может действовать как крупный бизнес, масштабы которого позволяют легко распространять вредоносные программы. У организаций есть партнёры, торговые посредники, поставщики и совладельцы, которые покупают и продают лицензии на вредоносное ПО другим преступным организациям для использования в новых регионах и на новых рынках.

Взломы, осуществляемые <<Чёрными шляпами>>, являются глобальной проблемой, которую крайне сложно решить. Работа 
правоохранительных органов осложняется тем, что хакеры оставляют мало улик, используют компьютеры ничего не подозревающих 
жертв и действуют в нескольких юрисдикциях. Иногда властям удаётся закрыть хакерский сайт в одной стране, однако эти же 
действия могут выполняться в другом месте, что позволяет преступной группе продолжать работу.

\textbf{В противоположность <<чёрным>> \ есть <<белые>> \  хакеры.} <<Белые шляпы>> "---  это специалисты по безопасности, которые выявляют недостатки в защите компьютерных систем и сетей с целью их устранения. Они работают на компании или являются независимыми 
подрядчиками. Их деятельность помогает предотвратить несанкционированный доступ к данным и снизить вероятность кибератак.
<<Чёрные шляпы>> получают незаконный доступ к системам со злыми намерениями и часто с целью личного обогащения. В отличие от 
них, <<Белые>> сотрудничают с компаниями и помогают им устранять слабые места в их системах, чтобы гарантировать 
безопасность данных.

\textbf{Но также существует что-то среднее между <<чёрными>> и <<белыми>>,} так называемые <<Серые шляпы>> "---  это хакеры, которые ищут 
уязвимости в системах без разрешения или ведома владельца. Они могут сообщать о найденных уязвимостях владельцу за небольшую 
плату.

Некоторые Серые шляпы считают, что их действия приносят пользу компаниям, но владельцы компаний редко ценят 
несанкционированные вторжения. Часто реальный мотив Серых шляп "---  продемонстрировать навыки и добиться известности.

Серые шляпы иногда нарушают законы и стандарты этики, но не имеют злого умысла, характерного для <<Чёрных шляп>>. В отличие от 
Белых шляп, они не связаны этическими правилами взлома или трудовым договором. Если организация не обращает внимания на их 
действия, они могут использовать обнаруженные уязвимости самостоятельно или рассказать о них другим хакерам\cite{kaspersky}.


\newpage
\subsection{Плюсы и минусы работы в области защиты информации.}
Плюсы:
\begin{enumerate}
    \item На специалистов по защите информации сейчас высокий спрос, с каждым годом количество различных угроз и кибератак 
    стремительно увеличивается, что делает специалистов этой сферы ещё более востребованными.
    \item Стремительные изменения в сфере информационных технологий позволяют специалистам получать бесценный опыт и 
    постоянное профессиональное развитие, постоянно обучаясь и повышая свою квалификацию.
    \item Благодаря высокой востребованности данной сферы, специалисты получают достаточно высокую заработную плату.
\end{enumerate}
 
Минусы:
\begin{enumerate}
    \item Работа под постоянным давлением и стрессом "---  нужно мгновенно реагировать на угрозы кибербезопасности и защищать 
    информацию от внешних угроз.
    \item Большая ответственность за сохранность и конфиденциальность большого количества информации.
    \item Высокие требования к профессиональным навыкам: работа в области защиты информации требует наличия 
    специализированных знаний и опыта, которые не всегда могут быть легко достигнуты.
\end{enumerate}