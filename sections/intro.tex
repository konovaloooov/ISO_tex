В данный момент времени наблюдается переход общества от индустриального типа к информационному, 
в котором информация становится все более важным и ценным ресурсом, соответственно возрастает и 
потребность её защиты. В условиях постоянно растущих угроз информационной безопасности, таких 
как кибератаки, утечки данных и несанкционированный доступ к информации, специалисты по защите 
информации становятся всё более востребованными в обеспечении безопасности организаций и 
государств.

Спрос на специалистов по кибербезопасности продолжает расти, поскольку мир становится всё более 
виртуальным. Все денежные расчёты, образование, общение, сфера услуг и многое другое постепенно 
переходят на онлайн-формат. 

Целью данного реферата является изучение основных особенностей профессии специалиста по защите 
информации, включая его обязанности, требования к квалификации, методы и средства защиты 
информации, а также перспективы развития этой профессии.

